pagina 9 picman

\section{Basic objects}
\begin{itemize}
\item boxe, caixa
\item circle, cercle
\item ellipse, 
\item line
	\subitem <->
\item arrow
\item arc
	\subitem cw, girar de laltre canto
\item spline
\item text
\item move, espai en blanc
\item same, repeteix l'ultima operació
\end{itemize}

\section{Direcció}
\begin{itemize}
\item up
\item down, el seguent element cap aball
\item left
\item right
	\begin{itemize}
	\item at last circle
	\item fa la magia d'estar en la mateixa posició
	\item exemples en: Descobrint els fills
	\end{itemize}
\end{itemize}

\section{Size}
\begin{itemize}
\item height #f (ht)
\item width #f (wid)
\item radius #f (rad)
\item diameter (diam)
\end{itemize}

\section{Personalitzant}
\begin{itemize}
\item dashed #f, discontinu
	\subitem line
	\subitem box
\item dotted, discontinu en punts
	\subitem line
	\subitem box
\item invis, si aixo xD
\item fill, enfosqueix
\end{itemize}

\section{Text}
\begin{itemize}
\item rjust, ajusta el text a la dreta
\item above, per sobre
\item below, per sota
\end{itemize}

\section{Definicions}
\begin{itemize}
\item reset; -> ho posa tot aixi
\item boxwid = .75
\item linewid =.75
\item circlerad = .25
\item ellipsewid = .75
\item movewid = .75
\item textwid = 0
\item arrowwid = .05
\item dashwid = .05
\item maxpsht = 8.5
\item fillval = .3
\item boxht = .5
\item lineht = .5
\item lineht = .5
\item arcrad = .25
\item ellipseht = .5
\item moveht = .5
\item textht = 0
\item arrowht = .1 /cap
\item arrowhead = 2 /diseny
\item maxpswid = 11 /dimensio maxima de les imatges
\item scale = 1
\end{itemize}
